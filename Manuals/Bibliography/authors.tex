\chapter{FDS Developers}

The Fire Dynamics Simulator and Smokeview are the products of an international collaborative effort led by
the National Institute of Standards and Technology (NIST) and VTT Technical Research Centre of Finland. Its developers and
contributors are listed below.

\vspace{0.3in}

\begin{flushleft}

Principal Developers of FDS  \\ [0.2in]

Kevin McGrattan, NIST, Gaithersburg, Maryland \\
Simo Hostikka, Aalto University, Espoo, Finland \\
Jason Floyd, Fire Safety Research Institute, UL Research Institutes, Columbia, Maryland \\
Randall McDermott, NIST, Gaithersburg, Maryland \\
Marcos Vanella, NIST, Gaithersburg, Maryland \\ [0.3in]

Principal Developer of Smokeview  \\ [0.2in]

Glenn Forney, NIST, Gaithersburg, Maryland \\ [0.3in]

Principal Developer of FDS+Evac  \\ [0.2in]

Timo Korhonen, VTT, Finland \\ [0.3in]

Contributors \\ [0.2in]

Salah Benkorichi, BB7, UK \\
Daniel Haarhoff, J\"ulich Supercomputing Centre, Germany \\
Susan Kilian, hhpberlin, Germany \\
Vivien Lecoustre, University of Maryland, College Park, Maryland \\
Anna Matala, VTT, Finland \\
William Mell, U.S. Forest Service, Seattle, Washington \\
Kristopher Overholt, RStudio, Austin, Texas \\
Benjamin Ralph, University of Edinburgh, UK \\
Topi Sikanen, VTT, Finland \\
Julio Cesar Silva, Brazilian Navy, Brazil \\
Ben Trettel, The University of Texas at Austin \\
Craig Weinschenk, UL Fire Safety Research Institute, Columbia, Maryland

\end{flushleft}


\chapter{About the Developers}

\begin{description}

\item[Kevin McGrattan] is a mathematician in the Fire Research Division of NIST. He received a bachelor of science degree from the School of Engineering and Applied Science of Columbia University in 1987 and a doctorate at the Courant Institute of New York University in 1991. He joined the NIST staff in 1992 and has since worked on the development of fire models, most notably the Fire Dynamics Simulator.

\item[Simo Hostikka] is an associate professor of fire safety engineering at Aalto University School of Engineering, since January 2014. Before joining Aalto, he worked as a Principal Scientist and Team Leader at VTT Technical Research Centre of Finland. He received a master of science (technology) degree in 1997 and a doctorate in 2008 from the Department of Engineering Physics and Mathematics of the Helsinki University of Technology.  He is the principal developer of the radiation and solid phase sub-models within FDS.

\item[Jason Floyd] is a Lead Research Engineer at the Underwriters Laboratories Fire Safety Research Institute in Columbia, Maryland. He received a B.S. (1993), M.S (1995), and a Ph.D. (2000) from the Nuclear Engineering Program of the University of Maryland. After graduating, he was awarded a National Research Council Post-Doctoral Fellowship at the Building and Fire Research Laboratory of NIST. He is a principal developer of the combustion, control logic, aerosol, droplet evaporation, and HVAC sub-models within FDS.

\item[Randall McDermott] joined the Fire Research Division at NIST in 2008. He received a B.S.~from the University of Tulsa in Chemical Engineering in 1994 and a Ph.D.~from the University of Utah in 2005. His research interests include subgrid-scale models and numerical methods for large-eddy simulation, turbulent combustion, immersed boundary methods, and Lagrangian particle methods.

\item[Marcos Vanella] joined the Fire Research Division at NIST in 2019. He received diplomas in Mechanical and Aeronautical Engineering from the National University of Cordoba, Argentina, and M.S.~and Ph.D.~degrees in Mechanical Engineering from the University of Maryland, College Park. His research interests include computer simulation and scientific software development applied to engineering systems, mainly in the areas of fluid flow and multiphysics interaction problems.

\item[Glenn Forney] is a computer scientist in the Fire Research Division of NIST.  He received a bachelor of science degree in mathematics from Salisbury State College and a master of science and a doctorate in mathematics from Clemson University.  He joined NIST in 1986 (then the National Bureau of Standards) and has since worked on developing tools that provide a better understanding of fire phenomena, most notably Smokeview, an advanced scientific software tool for visualizing Fire Dynamics Simulation data.

\item[Timo Korhonen] is a Senior Scientist at VTT Technical Research Centre of Finland. He received a master of science (technology) degree in 1992 and a doctorate in 1996 from the Department of Engineering Physics and Mathematics of the Helsinki University of Technology. He is the principal developer of the evacuation sub-model within FDS.

\item[Daniel Haarhoff] did his masters work at the J\"ulich Supercomputing Centre in Germany, graduating in 2015. His thesis is on providing and analyzing a hybrid parallelization of FDS. For this, he implemented OpenMP into FDS 6.

\item[Susan Kilian] is a mathematician with numerics and scientific computing expertise. She received her diploma from the University of Heidelberg and received her doctorate from the Technical University of Dortmund in 2002. Since 2007 she has been a research scientist for hhpberlin, a fire safety engineering firm located in Berlin, Germany. Her research interests include high performance computing and the development of efficient parallel solvers for the pressure Poisson equation.

\item[Vivien Lecoustre] is a Research Associate at the University of Maryland. He received a master of science in Aerospace Engineering from ENSMA (France) in 2005 and a doctorate in Mechanical Engineering from the University of Maryland in 2009. His research interests include radiation properties of fuels and numerical turbulent combustion.

\item[Anna Matala] worked as a research scientist at VTT Technical Research Centre of Finland 2008-2019. She received her PhD from Aalto University School of Science in 2013 and MSc in Systems and Operations Research from Helsinki University of Technology in 2008. She works as a fire safety engineering and research consultant. Her research concentrates on pyrolysis modelling and parameter estimation in fire simulations.

\item[William (Ruddy) Mell] is an applied mathematician currently at the U.S. Forest Service in Seattle, Washington. He holds a B.S. degree from the University of Minnesota (1981) and doctorate from the University of Washington (1994). His research interests include the development of large-eddy simulation methods and sub-models applicable to the physics of large fires in buildings, vegetation, and the wildland-urban interface.

\item[Kristopher Overholt] is a solutions engineer at RStudio. He received a B.S. in Fire Protection Engineering Technology from the University of Houston-Downtown in 2008, an M.S. in Fire Protection Engineering from Worcester Polytechnic Institute in 2010, and a Ph.D. in Civil Engineering from The University of Texas at Austin in 2013. He worked in the Fire Research Division at NIST from 2013 to 2015, where he was central to the development of the FDS continuous integration framework, Firebot.  He also worked on aspects of FDS related to verification and validation and quality metrics. His research interests include inverse fire modeling problems, soot deposition in fires, and the use of fire models in forensic applications.

\item[Topi Sikanen] is a Research Scientist at VTT Technical Research Centre of Finland and a graduate student at Aalto University School of Science. He received his M.Sc.~degree in Systems and Operations Research from Helsinki University of Technology in 2008. He works on the Lagrangian particle and liquid evaporation models.

\item[Ben Trettel] is a graduate student at The University of Texas at Austin. He received a B.S.~in Mechanical Engineering in 2011 and an M.S.~in Fire Protection Engineering in 2013, both from the University of Maryland. He develops models for the transport of Lagrangian particles for the Fire Dynamics Simulator.

\item[Julio Cesar Silva] is a Lieutenant in the Naval Engineers Corps of the Brazilian Navy. He worked in the Fire Research Division of NIST as a Guest Researcher from National Council for Scientific and Technological Development, Brazil. He received a M.Sc.~in 2010 and a doctorate in 2014 from Federal University of Rio de Janeiro in Civil Engineering. His research interests include fire-structure interaction and he develops coupling strategies between FDS and finite-element codes.

\item[Benjamin Ralph] is a fire safety engineer at Foster + Partners. He received his M.Eng.~in Civil Engineering from the University of Southampton,~UK in 2008, P.G.Dip.~in Fire Safety Engineering from the University of Ulster,~UK in 2014, and Ph.D.~in Fire Safety Science from the University of Edinburgh,~UK in 2018. He was a Guest Researcher in the Engineered Fire Safety Group at NIST in 2016. His research interests are related to performance-based design and he is a developer of the HVAC sub-model - specifically the transient mass and energy transport solver.

\item[Salah Benkorichi] is a researcher and a Fire Engineer at the BB7, in Manchester, UK. He received his M.Sc.~in 2016 from the University of Poitiers. His research activities focus on flame spread and pyrolysis modeling using multi-scale methods.

\item[Craig Weinschenk] is a Lead Research Engineer at the Underwriters Laboratories Fire Safety Research Institute, in Columbia, Maryland. He worked in the Fire Research Division at NIST as a National Research Council Postdoctoral Research Associate in 2011. He received a B.S.~from Rowan University in 2006 in Mechanical Engineering. He received an M.S.~in 2007 and a doctorate in 2011 from The University of Texas at Austin in Mechanical Engineering. His research interests include numerical combustion, fire-structure interaction, and human factors research of fire-fighting tactics.

\end{description}



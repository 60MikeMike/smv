\subsection{Terrain File Format}

Terrain files are unformatted and are output by the Wildland-Urban
interface (WUI) version of FDS.  The terrain file header is
written out using:
\begin{lstlisting}
      WRITE(IUNIT) 1
      WRITE(IUNIT) VERSION
      WRITE(IUNIT) XMIN, XMAX, YMIN, YMAX
      WRITE(IUNIT) NX, NY
      WRITE(IUNIT) (ZELEV(I,J),I=1,NX),J=1,NY)
      WRITE(IUNIT) (STATE(I,J),I=1,NX),J=1,NY)
\end{lstlisting}
The first ``1'' is used by Smokeview to determine whether the file
was written on a big or little endian computer.  {\tt VERSION} is
an integer value representing the file version (initially 1), {\tt
XMIN, XMAX, YMIN, YMAX} are floating point values defining the
terrain boundary, {\tt NX} and {\tt NY} are integers defining the
number of cells in the $x$ and $y$ direction, {\tt ZELEV} are
floating point values defining the elevation at each grid cell and
{\tt STATE} are one byte integers defining the states at each grid
cell.

The WUI \fds\ software outputs terrain frames at fixed time
intervals but there is no requirement by \smokeview\ for this to
happen. Because, cell states change infrequently, the cell states
are only output when they change.  Each terrain frame is written
using:
\begin{lstlisting}
      WRITE(IUNIT) TIME
      WRITE(IUNIT) NCHANGES
      WRITE(IUNIT) (CELLINDEX(I),I=1,NCHANGES)
      WRITE(IUNIT) (CELLSTATES(I),I=1,NCHANGES)
\end{lstlisting}
where {\tt TIME} is the time in seconds when the data is output,
CELLINDEX are the cell indices ($cellindex=(J-1)*NY + I - 1$) of
the cells that have changed state and CELLSTATES are the value of
the changed state.  The data for the very first time step (at
$t=0.0$) is not output explicitly but is assumed from information
previously output in the header.
